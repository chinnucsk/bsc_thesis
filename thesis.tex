\documentclass[12pt, a4paper, oneside]{book}
\usepackage[left=3.5cm, top=2.5cm, right=2.5cm, bottom=2.5cm]{geometry}
\usepackage[utf8]{inputenc}
\usepackage[T1]{fontenc}
\usepackage[english,hungarian]{babel}
\selectlanguage{hungarian}
\usepackage{fancyhdr}
\usepackage{setspace}
\usepackage{listings}
\usepackage{makeidx}
\usepackage{natbib}
\usepackage{url}

\makeindex

\newenvironment{abstract}
{\newpage \pagestyle{empty} \vspace*{\fill} \begin{center}\em{Absztrakt}\end{center}}
{\vspace*{\fill} \newpage}

\pagestyle{fancy}
\renewcommand{\chaptermark}[1]{%
\markboth{\thechapter.%
\ \chaptername.\ #1}{}}

\fancyhead[L]{}


%\renewcommand\lstlistingname{ábra}
%\renewcommand\lstlistlistingname{Ábrák}
\lstset{caption=\thelstlisting. \lstlistingname}


\author{Czinkos Zsolt}
\title{Erlang/OTP: magas rendelkezésre állású elosztott rendszerek fejlesztése}
\date{2012-04-28}

\begin{document}
\maketitle

\onehalfspacing

\begin{abstract}
Az Erlang programozási nyelvet az Ericssonnál hozták létre hálózati
eszközök, telefonrendszerek programozására. A magas telekom elvárások
tükröződnek az Erlangban fejlesztett rendszerek architektúrájában,
amely az Open Telecom Platform szoftverkönyvtárban ölt testet. A dolgozat
bemutatja az Erlang programozási nyelvet, a fejlesztési elveket és
alkalmazási lehetőségeit a webes technológiára épülő alkalmazások területén.
\end{abstract}

\newpage
\pagestyle{empty}
\vspace*{\fill} 
\hfill \emph{Mindenkinek, aki szereti}
\vspace*{\fill} 
\newpage

\pagestyle{fancy}

\tableofcontents

\chapter{Bevezetés}

Az Internet -- azon belül különösen a Web -- terjedésével párhuzamosan nőtt az
igény kiszámítható, jó minőségű szolgáltatásokra. A szolgáltatóknak egyre
magasabb elvárásoknak kell megfelelnie -- nem utolsó sorban azért, mert a
felhasználók fejében a web és az ingyenes tartalom összeforrt. Még színvonalas
termékekért is nehezen adnak ki pénzt, nemhogy hibás, elavult tartalomért,
akadozó és kiszámíthatatlanul működő szolgáltatásokért. Ma már a hálózat nem
csupán mérnököknek, kutatóknak érdekes kábelezést jelent, amely így-úgy hasznos
a tudományos kutatásaik során, hanem a mindennapi élet részét képező társadalmi
kapcsolatok \textit{reprezentációját} is. A felhasználó egyre aktívabb
\textit{cselekvője} ezeknek a valós vagy virtuális világban létrejött
hálózatoknak, egyre inkább itt keresi (és többnyire találja meg) azt a teret,
ahol ismerik, és ő is ismer, ahol ura annak az eszköztárnak, amelynek
birtokában különböző -- rövid, prompt, aszinkron, szöveg, hang vagy videó --
\textit{üzenetek} segítségével ápolni tudja kapcsolatait. Ez a
kapcsolatrendszer és eszköztár jelenti azt az új mikroikozmoszt, amelyben a
felhasználó -- cselekvő- és befolyásolóképessége tudatában -- kényelemben és
biztonságban érzi magát.

Ez a kényelem és biztonság \textit{függővé} tesz: világunk megszokott
működésének zavarait nehezen vagy egyáltalán nem tudjuk tolerálni,
kiszolgáltatottnak és tehetetlennek érezzük magunkat. Ilyenkor derül ki, hogy
bár mikrokozmoszunkat ismerni véljük, az azt működtető rendszer elemeit meg sem
tudjuk nevezni, csak azt tudjuk, hogy \glqq{}van\textquotedblright{} (ez a
valami pillanatnyilag a legtöbb ember számára néhány cég szolgáltatásában ölt
testet: Facebook, Google, Twitter). A szoftverfejlesztőknek, tervezőknek ennek
a világnak a működtetéséhez szükséges rendszert kell tudniuk megépíteni és
üzemeltetni úgy, hogy a felhasználók a lehető legkevesebb alkalommal
szembesüljenek azzal, hogy kihúzták alóluk a talajt. Nem emberbaráti, hanem
üzleti megfontolások miatt.

A weben a sikerhez elengedhetetlen a folyamatos és megbízható szolgáltatás,
nagyon alacsony az ingerküszöb, ha egy oldal betöltődése tovább tart mint 4
másodperc, már odébb is állt a felhasználó \citep{AkamaiReport}.  Ha túl
sokszor kap hibaüzenetet -- amitől jobb esetben ingerült lesz, rosszabb esetben
halálra rémül, hátha ő rontott el valamit --, keres mást. Éppen ezért nagyon
fontos, hogy olyan rendszert építsünk, amely 

\begin{enumerate} 
  \item folyamatosan, megszakítás nélkül működik;
  \item megfelelő válaszidővel, sebességgel működik;
  \item funkcionálisan jól működik;
  \item a felmerülő hibák nyomon követhetők, kezelhetők.
\end{enumerate}

A fentebb már említett vezető webes cégek mind megfelelnek ezeknek a
követelményeknek, persze nem kevés munka és pénz árán. A felhasználót azonban a
legkevésbé sem érdekli, hogy a szolgáltatást nyújtó üzleti vállalkozás hogyan
tudja működtetni rendszerét, hány embert alkalmaz, stb. Őt az érdekli, hogy
neki ingyen vagy elérhető áron a lehető legtöbbet nyújtsa. Ez az elvárása
sajnos (vagy szerencsére) nem csak a mammutcégekkel szemben áll fent, hanem
minden webes céggel szemben, mindenhol szeretné megkapni azt a minőséget,
amihez hozzászokott. Azt a céget tekinti profinak, jónak, amely ugyanazt tudja
nyújtani. Ha egy cég sikert akar, akkor már induláskor fel kell készülnie arra,
hogy ha elsül a kapanyél, és özönlenek a felhasználók, akkor tartani tudja az
iramot, ki tudja szolgálni ugrásszerűen megnőtt ügyfélkörét; miközben egy kezdő
vállalkozás nem engedhet meg magának földrajzilag diverzifikált többtízezer
gépes szerverparkot: kicsiből indulva kell képesnek lennie a növekedésre.

Hogyan lehet olyan rendszert építeni, amellyel neki lehet vágni egy webes
vállalkozásnak úgy, hogy ne kelljen attól félni, mi lesz, ha holnap regisztrál
még 10 ezer felhasználó (4 másodperc!), vagy ha tönkremegy az egyik gép?

Számos programozási nyelv és környezet közül lehet ma már választani, amely
alkalmas erre a feladatra, ez a dolgozat az Erlang programozási nyelvet és a
hozzá kapcsolódó Open Telecom Platform-ot (OTP) mutatja be. Az Erlang egy
funkcionális programozási nyelv, amelyet az Ericsson fejlesztett ki mintegy 20
évvel ezelőtt telefonrendszerek, szoftveres kapcsolóközpontok programozásához,
a telekommunikációs iparban szokásos rendkívül magas elvárásoknak megfelelően. 

Az Erlang megalkotásánál az elsődleges cél magas rendelkezésre állású
(\emph{highly available}), hibatűrő (\emph{fault tolerant}) redundáns
rendszerek építése volt. Ez az, amire az Erlang igazán alkalmas, ez az a
terület, ahol az Erlangnak évtizedes múltja van: akár 99,999\%-os
rendelkezésre állás biztosításában. Az hozzávetőleg 5 perc kiesés
évente \citep{WikipediaNines}.

1998-ban open source-szá vált a nyelv és a platformot adó
szoftverkönyvtárak, azóta bárki használhatja bármilyen feladatra, számos
önkéntes és cég teszi be a közösbe a maga alkalmazását: HTTP szervert, NoSQL
adatbáziskezelőt, stb.

\newpage

\chapter{Elméleti alapok}
\section{Funkcionális programozás}
\section{Aktor modell}
\section{Az Erlang programozási nyelv}
\section{Open Telecom Platform}
Behaviours. Supervisor hierarchies.
\section{Egy gép nem gép: elosztott rendszerek}
Erlang nodes. CAP theorem.

\chapter{Soft realtime messaging -- a simple demo}

\chapter{A messaging szerver}
\section{Felépítés}
\section{Hibatűrés}
\section{Elosztott rendszer -- kell a redundancia}

\chapter{Web interfész}
\section{RESTful webservices}
\section{HTML5 -- websockets}

\chapter{Üzemeltetés, karbantartás}
\section{Naplózás}
Event manager, event handler.
\section{Hibakeresés}
Trace.
\section{Hibajavítás, verzióléptetés}
Upgrade.
\section{Szállítás}

\chapter{Alkalmazási lehetőségek, kitekintés}

\chapter{Összegzés}
\appendix

{\footnotesize

\begin{lstlisting}[emph={fun}, emphstyle=\textbf, caption=Map reduce module, language=erlang, frame=single]
-module(mapreduce).

start() ->
  spawn(fun() -> init() end).

init() ->
  loop().

loop() ->
  receive
    {From, To, What} ->
      io:format("~p sent to ~p a message ~p~n", [From,To,What]),
      loop();
    _ -> % avoid full msg box
      io:format("Nothing. Finish.")
  end.

\end{lstlisting}
}

\begin{thebibliography}{9}

\harvarditem[Armstrong]{}{2003}{ArmstrongThesis}
Armstrong, Joe (2003): \emph{Making reliable distributed systems in the presence of software errors}. PhD.
thesis, The Royal Institute of Technology Stockholm, Sweden. Web:
\url{http://www.erlang.org/download/armstrong_thesis_2003.pdf}, letöltés dátuma: 2012-04-01

\harvarditem[Armstrong]{}{2007}{ArmstrongBook}
Armstrong, Joe (2007): \emph{Programming Erlang: Software for a Concurrent World}. USA: The Pragmatic
Bookshelf.

\harvarditem[Cesarini]{}{2009}{CesariniBook}
Cesarini, Francesco – Thomson, Simon (2009): \emph{Erlang programming}. USA: O'Reilly Media.

\harvarditem[Logan et al.]{}{2011}{OTPInAction}
Logan, Martin – Merritt, Eric – Carlsson, Richard (2011): \emph{Erlang and OTP in Action}. USA:
Manning Publications.

\harvarditem[Fielding]{}{2001}{Fielding}
Fielding, Roy Thomas (2001): \emph{Architectural Styles and the Design of Network-based Software
Architectures}. PhD. thesis, University of California, Irvine. Web:
\url{http://www.ics.uci.edu/~fielding/pubs/dissertation/fielding_dissertation.pdf}, letöltés dátuma: 2012-04-
01.

\harvarditem[Erlang documentáció]{}{2012}{ErlangDoc}
Erlang documentation... (2011) Web: \url{http://www.erlang.org}

\harvarditem[Akamai felmérés]{}{2006}{AkamaiReport} 
Akamai felmérés (2006): \emph{Retail web site performance: Consumer Reaction to a Poor Online
Shopping Experience}
\url{http://www.akamai.com/dl/reports/Site_Abandonment_Final_Report.pdf}

\end{thebibliography}

\clearpage
\addcontentsline{toc}{chapter}{Tárgymutató}
\printindex

\end{document}
